\documentclass[a4paper,10pt,fleqn]{article}

\usepackage[a4paper]{geometry}
\geometry{verbose,tmargin=2.5cm,bmargin=2.5cm,lmargin=2.5cm,rmargin=2.5cm}

\setlength{\parskip}{\smallskipamount}
\setlength{\parindent}{0pt}

\usepackage{amsmath}
\usepackage{amssymb}
\usepackage{braket}

\usepackage[libertine]{newtxmath}
\usepackage[no-math]{fontspec}
\setmainfont{Linux Libertine O}
\setmonofont{JuliaMono-Regular}

\usepackage{hyperref}
\usepackage{url}
\usepackage{xcolor}

\usepackage[normalem]{ulem}

\usepackage{mhchem}

\usepackage{minted}
\newminted{julia}{breaklines,fontsize=\footnotesize}
\newminted{text}{breaklines,fontsize=\footnotesize}


\newcommand{\txtinline}[1]{\mintinline[fontsize=\footnotesize]{text}{#1}}
\newcommand{\jlinline}[1]{\mintinline[fontsize=\footnotesize]{julia}{#1}}

\definecolor{mintedbg}{rgb}{0.90,0.90,0.90}
\usepackage{mdframed}

%\BeforeBeginEnvironment{minted}{\begin{mdframed}[backgroundcolor=mintedbg,%
%  rightline=false,leftline=false,topline=false,bottomline=false]}
%\AfterEndEnvironment{minted}{\end{mdframed}}


\usepackage{tikz}
\usetikzlibrary{shapes.geometric}
\tikzstyle{mybox} = [draw=blue, fill=green!5, very thick, rectangle,
  rounded corners, inner sep=10pt, inner ysep=20pt]
\tikzstyle{fancytitle} = [fill=blue, text=white]

\renewcommand{\imath}{\mathrm{i}}

\begin{document}

\title{\textsf{PWDFT.jl} Documentation}
\author{Fadjar Fathurrahman}
\date{}
\maketitle

%\tableofcontents

\textbf{This document is a work in progress}

In this part I will describe my design choices in implementing \textsf{PWDFT.jl}.
This design is by no means perfect
and it might change in the future to accomodate more complex use cases.

\section{Overview}

The design of \textsf{PWDFT.jl} is intended to be rather simple. One constraint
that is set to the code is that it should be possible to perform application
of Hamiltonian operator to wave function as simple as:
%
\begin{juliacode}
Hpsi = Ham*psi # or
Hpsi = op_H(Ham, psi)
\end{juliacode}
%
where \jlinline{psi} is, currently, of type \jlinline{Array{ComplexF64,2}}
\footnote{This function may be extended take other types other that plain Julia
array for more complex case.}.
%
This comes with an important consequences: all other pieces of information
about how this operation is done should be present in the type of \jlinline{Ham}.
\footnote{We will also see some quirks related to this design choice later,
such as applying Hamiltonian to several k-points or spin-polarized case}.

In \textsf{PWDFT.jl}, the type of \jlinline{Ham} is \jlinline{Hamiltonian}.
Several important fields of \jlinline{Hamiltonian} are instances of the following
types (please refer to the source code for more details about this):
\begin{itemize}
\item \jlinline{Atoms}: contains information about atomic structure: cell
vectors, atomic species and atomic coordinates.
\item \jlinline{PsPot_GTH}: contains information about atomic pseudopotentials.
\item \jlinline{Electrons}: contains information about electronic states.
\item \jlinline{PWGrid}: contains information about plane wave basis set.
\item \jlinline{Potentials}: contains information about local potentials such
as local pseudopotential, Hartree and exchange-correlation potential.
\item \jlinline{PsPotNL}: contains information about nonlocal pseudopotential
terms.
\item \jlinline{Energies}: contains information about components of Kohn-Sham
energy.
\item \jlinline{SymmetryInfo}: contains information about symmetry operations.
\end{itemize}


\input{Atoms}

\input{PWGrid}

\input{Electrons}

\input{PotentialsEnergies}

\input{Pseudopotential}

\input{Hamiltonian}

\input{SCF}

\input{DirectMin}

\section{Calculation of atomic forces}

These notes are adapted from Kohanoff2006.

The forces on the nuclei of atomic species $s$ are given by
\begin{equation}
\mathbf{F}_{I}^{s} = -\frac{\partial E_{\mathrm{KS}}}{\partial\mathbf{R}_{I}^{s}}
\end{equation}
Specifically:
\begin{equation}
\mathbf{F}_{I}^{s} = \int \mathbf{\rho}(\mathbf{r})\left(V^{\mathrm{Ps,loc}}_{s}\right)'
\left(\left| \mathbf{r} - \mathbf{R}_{I} \right|\right)
\frac{\mathbf{r} - \mathbf{R}_{I}}{\left| \mathbf{r} - \mathbf{R}_{I} \right|}
\,\mathrm{d}\mathbf{r} +
\frac{Z_{I}}{2} \sum_{J \neq I} Z_{J}
\frac{\mathbf{R}_{I} - \mathbf{R}_{J}}{\left| \mathbf{R}_{I} - \mathbf{R}_{J} \right|^3} +
\mathbf{F}^{\mathrm{Ps,nloc}}_{s,I}
\end{equation}

$\left( V^{\mathrm{Ps,loc}}_{s} \right)'$ is the derivative of the local part of the pseudopotential

$\mathbf{F}_{\mathrm{Ps,nloc}}^{s}$: nonlocal pseudopotential contribution to the force

Using plane wave basis set.

Nuclear-nuclear contribution:
\begin{equation}
F^{NN}_{s,I} = \frac{Z_{I}}{2}\sum_{J \neq I}^{P} Z_{J}
\sum_{L=-L_{\mathrm{max}}}^{L_{\mathrm{max}}}
\left( \mathbf{R}_{I} + l - \mathbf{R}_{J} \right) \times
\left[
\frac{\mathrm{erfc}\left(\left|\mathbf{R}_{I} + l - \mathbf{R}_{J} \right|\eta\right)}{\left|\mathbf{R}_{I} + l - \mathbf{R}_{J} \right|^3} +
\frac{\eta \mathrm{e}^{-\eta^2}\left|\mathbf{R}_{I} + l - \mathbf{R}_{J} \right|^2}
{\left|\mathbf{R}_{I} + l - \mathbf{R}_{J} \right|}
\right]
\end{equation}

Local pseudopotential contribution:
\begin{equation}
F^{\mathrm{Ps,loc}}_{s,I} = \Omega \sum_{\mathbf{G} \neq \mathbf{0}}
\mathrm{i}\mathbf{G}\,\mathrm{e}^{\mathbf{G}\cdot\mathbf{R}_{I}}
V^{\mathrm{Ps,loc}}_{\mathrm{s}}(G) \rho(\mathbf{G})
\end{equation}

Nonlocal pseudopotential contribution:
\begin{equation}
F^{\mathrm{Ps,nloc}}_{s,I} = 2\mathrm{Re}\left\{
\sum_{\mathbf{k}} \omega_{\mathbf{k}} \sum_{lm} \sum_{i=1}^{N_{\mathbf{k}}}
f^{(\mathbf{k})}_{i}
\alpha^{s}_{lm} F^{lm,s*}_{I,i} D^{lm,s*}_{I,i}
\right\}
\end{equation}
where
\begin{equation}
D^{lm,s*}_{I,i} = \sum_{\mathbf{G}} \imath \mathbf{G}
e^{\imath\mathbf{G}\cdot\mathbf{R}_{I}}
f^{s}_{lm}(\mathbf{G}+\mathbf{k}) C_{i\mathbf{k}}(\mathbf{G})
\end{equation}

\appendix
\section{Howtos}

This part contains miscellaneous info.

TODO: Some of the should be moved into main text.

\input{HOWTOS.tex}


\section*{Status}

\textbf{29 July 2019} Total energy results are now similar to ABINIT
and Quantum ESPRESSO. A rather comprehensive test has been added
for SCF and Emin PCG for several simple systems.


\textbf{28 May 2018} The following features are working now:
\begin{itemize}
\item LDA and GGA, spin-paired and spin polarized calculations
\item Calculation with k-points (for periodic solids).
  \textsf{SPGLIB} is used to reduce the Monkhorst-Pack grid points
  for integration over Brillouin zone.
\end{itemize}

Band structure calculation is possible in principle as this can be
done by simply solving
Schrodinger equation with converged Kohn-Sham potentials, however there
is currently no tidy script or function to do that.

Total energy result for isolated systems (atoms and molecules) agrees quite
well with ABINIT and PWSCF results.

\sout{Total energy result for periodic solid is quite different from ABINIT and PWSCF.
I suspect that this is related to treatment of electrostatic terms in periodic system.}

These discrepancies have been minimized. For several systems the agreement is very good
even though I did not use the same algorithm as ABINIT.

\sout{SCF is rather shaky for several systems, however it is working in quite well in nonmetallic
system.}

SCF stability has been improved with Pulay mixing and its variants.



\bibliographystyle{unsrt}
\bibliography{BIBLIO}



\end{document}
